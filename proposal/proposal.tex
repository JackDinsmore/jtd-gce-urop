\documentclass[11pt]{article}
\title{Refining the Millisecond Pulsar Model for the Galactic Center Excess}
\author{\Large{UROP Proposal with Tracy Slatyer}\vspace{12pt}\\Jack Dinsmore}


\usepackage{csquotes}
\usepackage[english]{babel}
\usepackage[left=1in, right=1in, top=1in, bottom=1in]{geometry}



\begin{document}
\maketitle

\begin{displayquote}
    I am a Junior in the MIT Physics department, applying for a direct-funding astrophysics UROP with Professor Tracy Slatyer of MIT in the fall of 2020. I will be working remotely from Leverett, MA.
\end{displayquote}

\section{Project Description}
The Galactic Center GeV Excess (GCE) is an unexpected source of gamma radiation originating from the center of the Milky Way, detected  recently by the \textit{Fermi} Gamma-ray Space Telescope \cite{fermilab, tracy3}. Its origin is debated; the GCE holds potential to be the first evidence observed for dark matter annihilation, yet several studies have also shown that point sources such as millisecond pulsars may be responsible for the excess \cite{bartels, tracy1}.

The recently collected 4FGL point source catalog promises to shed light on this conflict. By it, Ref. \cite{fermilab} argues that the millisecond pulsar model outlined in Ref. \cite{bartels} may not contribute to the GCE, and that if the GCE were entirely explained by pulsars, a large number of faint pulsars would be required. However, Ref. \cite{fermilab} uses an approximate luminosity function to describe pulsar brightness, and it is not known to what degree this approximation influences the results.

We intend to expand on Refs. \cite{bartels} and \cite{fermilab} in their investigations into a pulsar-based explanation for the GCE. In particular, we will test luminosity functions  more complicated than those used in \cite{fermilab}, taken from observed pulsars in other contexts, and measure how the number of pulsars required to recreate the GCE changes based on the luminosity functions used. We can then predict how many pulsars should have already been observed by other means for each luminosity function.


\section{Personal Role}
As the only student working on this project currently, I will be involved in all of the steps listed above. I will start by continuing to read literature on the GCE and on millisecond pulsars, identifying luminosity functions that have been observed for non-GCE scenarios. Then I will calculate the number of pulsars required to reproduce the GCE for each luminosity function, and compute how many should have been observed already.

For this first term of the project, I will probably spend most of my time gathering the luminosity functions and starting to reproduce the GCE, saving simulations and data analysis for a later term.


\section{Personal Statement}
I have been interested in astrophysics for most of my life, but the actual process of research is still new to me. I have learned from previous research projects---ranging from theoretical studies of Schwarzschild de Sitter black holes, to creating data analysis algorithms for use in the Large Hadron Collider, to analyses of time-indexed images of stellar clusters---that I am most drawn to projects with a foot in both worlds: theoretically-motivated, but backed up by observation. This UROP fits this category well.

It is also rare to find an undergraduate research opportunity so closely related to the most fundamental laws of physics. Some develop a new data analysis tool; some compile data points to add to an astrophysical catalog. But this UROP is a step in the long walk to understanding dark matter, one of the most mysterious substances remaining in the universe. I have never been involved in a project like this before, and I am excited to get started.

\section{Personal Goals}
Aside from the experience of working on a half-theoretical, half-observational research project, and aiding the effort to understand dark matter, I am looking forward to better understanding what professional research is like. I intend to apply to grad school, and I would like to be as sure as possible that research is what I want to do for a living before I graduate MIT.

I would also like to do a senior thesis next year. This project may or may not become that thesis project, but if not, it would certainly develop experience that I would use for my thesis.



\begin{thebibliography}{99}
    \bibitem{fermilab}Y. Zhong, S. McDermott, I. Cholis, P. Fox. ``A New Mask for An Old Suspect: Testing the Sensitivity of the Galactic Center Excess to the Point Source Mask.'' arXiv:1911.12369v (2019).


    
    \bibitem{tracy3}R. Leane, T. Slatyer. ``Spurious Point Source Signals in the Galactic Center Excess.'' arXiv:2002.12370v1 (2020).


    
    \bibitem{bartels}R. Bartels, S. Krishnamurthy, G. Weniger. ``Strong Support for the Millisecond Pulsar Origin of the Galactic Center GeV Excess.'' \textit{Phys. Rev. Lett} (2016).


    
    \bibitem{tracy1}S. Lee, M. Lisanti, B Safdi, T. Slatyer, W. Xue. ``Evidence for Unresolved $\gamma$-Ray Point Sources in the Inner Galaxy.'' \textit{Phys. Rev. Lett} (2016).


    
    \bibitem{tracy2}R. Leane, T. Slatyer. ``Dark Matter Strikes Back at the Galactic Center.'' arXiv:1904.08430v1 (2019).


\end{thebibliography}


\end{document}